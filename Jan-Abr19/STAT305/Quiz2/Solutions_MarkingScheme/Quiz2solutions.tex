\documentclass[20pt]{report}
\usepackage[left=1in, right=1in, top=1in, bottom=1in]{geometry}
\usepackage{mathexam}
\usepackage{amsmath}
\usepackage{xcolor}
\usepackage{hyperref}

\newcommand{\g}{\Gamma}
\newcommand{\gt}{\theta}
\newcommand{\ttil}{\tilde{\theta}}


\ExamClass{}
\ExamName{STAT305 2018-2019 Winter 2 - Quiz 2 Solutions (by Gio)}
\ExamHead{\today}

\let\ds\displaystyle

\begin{document}

\textbf{Important:}

I'm using a shorter notation for some items. Let me know if you have questions about the material. Questions about marking can be done by following the instructions here (TBA) \textbf{after} I hand back your quizzes. Please don't post questions related to marking on this post. \\

\textbf{Marking scheme:}

For questions worth \textbf{1 mark}: Either right or wrong as they usually require short and straightforward answers.

For questions worth \textbf{2 marks}: Solution with small mistakes, but sensitive answer and logical reasoning will be generally awarded [1] mark.

For questions worth \textbf{3 marks or more}: I detailed the marking scheme for those in the solutions below. You can find the marks in this format "[ mark ]".\\



\textbf{Solutions:}

a) This question basically asks to show that $Beta(1,\theta) = \frac{1}{\theta} $. Notice that
$ \Gamma(a) = (a-1)\Gamma(a-1) $. Therefore: \\

$ B(1, \theta) = \frac{\Gamma(1)\Gamma(\theta)}{\g(\theta + 1)} = \frac{0!\g(\theta)}{\theta\g(\theta)} = \frac{1}{\theta}
$ \\

\textbf{Remark: }$\theta!$ does not make sense as $\theta$ is not necessarily an integer. However, we will not deduct marks for that this time. \\


b)
$ 
L(\theta) = \theta^n (\prod y_i)^{\theta-1} 
$\\

c) 
$
log(L(\theta)) = nlog{\theta} + (\theta-1)log(\prod y_i) = 
nlog{\theta} + (\theta-1)\sum log(y_i)
$ \\

d) I will simplify notation a bit and use simple derivative notation as the questions uses $L(\theta)$: \\
$
dl/d\theta = \frac {n}{\theta} + \sum log(y_i)
$ \\

Setting the derivative equal to 0: \\
$
\hat{\theta} = \frac{n}{- \sum log(y_i)}
$ \textbf{[2]}\textbf{ [-1 if $\theta$ instead of $\hat{\theta}$] }\\

$
d^2l/d\theta^2 = \frac{-n}{\theta^2} <0
$ for all $\theta$. In particular, $d^2l/d\theta^2 < 0$ at $\theta = \hat{\theta} $ \textbf{[1] }\\

e) We don't know the distribution of $\tilde{\theta} $ and
$Var(\tilde{\theta})$ is not a linear combination of $Y_i$. [Either is OK] \\

f)$
- d^2l/d\theta^2 = \frac{n}{\theta^2} 
$ \\

g)$
\frac{\theta^2}{n}
$ \\

h) (Non-exact) Approximation. Justification won't be marked, but check question j for the exact expression for $Var(\tilde{\theta})$. \\

i)\\ 

i. See table for pdf of $Gamma(n,\theta)$. \\

ii. See my post with tittle "Gamma properties" on piazza: (The answer is on page 2. We also discussed it during office hours and labs.)\\

iii. $E(\ttil) = \frac{n\gt}{n-1}$ \\

iv. No. $E(\ttil) \neq \theta $ \\

\pagebreak

j) A sufficient condition for an estimator to be consistent is: \\

$
MSE(\ttil) = Bias^2(\ttil) + Var(\ttil) \rightarrow 0
$ as n increases. \textbf{[1]} \\

From the question, $ Var(\ttil) = o(1/n) \rightarrow 0 $. \textbf{[1]} \\

Additionally, $ Bias^2(\ttil)= (\frac{n\gt}{n-1} - \theta)^2 \rightarrow 0 $ 
(i.e, $\ttil$ is not unbiased, but is asymptotically unbiased: $ E(\ttil) = \frac{n\gt}{n-1} \rightarrow \theta$) \textbf{[1]} \\

k) \textbf{(Bonus)} Following the hint: \\

$F_{X_i}(x) = P(X_i \leq x) = P (X_i < x) = P(-log(Y_i) < x) = P(log(Y_i) > -x) = 
P(Y_i > e^{-x}) = \\ 1 - P(Y_i \leq e^{-x}) = 1 - F_{Y_i} (e^{-x})$ \textbf{[1]} \\

$ F_{Y_i}(y) = \int_0^y \theta t^{\theta-1} dt =  y^\theta
$ \\

Therefore:
$ F_{X_i}(x) = 1 - (e^{-x}) ^\theta = 
1 - e^{-\theta x} $, which is the CDF of Exp($\theta$) \textbf{[1]} \\

Finally, since $X = \sum_{i=1}^n X_i$, $X \sim Gamma(n,\theta)$ as required. \textbf{[1]}

\end{document}